%%%%%%%%%%%%%%%%%
% This is an sample CV template created using altacv.cls
% (v1.3, 10 May 2020) written by LianTze Lim (liantze@gmail.com). Now compiles with pdfLaTeX, XeLaTeX and LuaLaTeX.
%
%% It may be distributed and/or modified under the
%% conditions of the LaTeX Project Public License, either version 1.3
%% of this license or (at your option) any later version.
%% The latest version of this license is in
%%    http://www.latex-project.org/lppl.txt
%% and version 1.3 or later is part of all distributions of LaTeX
%% version 2003/12/01 or later.
%%%%%%%%%%%%%%%%

%% If you need to pass whatever options to xcolor
\PassOptionsToPackage{dvipsnames}{xcolor}

%% If you are using \orcid or academicons
%% icons, make sure you have the academicons
%% option here, and compile with XeLaTeX
%% or LuaLaTeX.
% \documentclass[10pt,a4paper,academicons]{altacv}

%% Use the "normalphoto" option if you want a normal photo instead of cropped to a circle
% \documentclass[10pt,a4paper,normalphoto]{altacv}

\documentclass[10pt,a4paper,ragged2e,withhyper]{altacv}

%% AltaCV uses the fontawesome5 and academicons fonts
%% and packages.
%% See http://texdoc.net/pkg/fontawesome5 and http://texdoc.net/pkg/academicons for full list of symbols. You MUST compile with XeLaTeX or LuaLaTeX if you want to use academicons.

% Change the page layout if you need to
\geometry{left=1.25cm,right=1.25cm,top=1.5cm,bottom=1.5cm,columnsep=1.2cm}

% The paracol package lets you typeset columns of text in parallel
\usepackage{paracol}

% Change the font if you want to, depending on whether
% you're using pdflatex or xelatex/lualatex
\ifxetexorluatex
  % If using xelatex or lualatex:
  \setmainfont{Roboto Slab}
  \setsansfont{Lato}
  \renewcommand{\familydefault}{\sfdefault}
\else
  % If using pdflatex:
  \usepackage[rm]{roboto}
  \usepackage[defaultsans]{lato}
  % \usepackage{sourcesanspro}
  \renewcommand{\familydefault}{\sfdefault}
\fi

% Change the colours if you want to
\definecolor{SlateGrey}{HTML}{2E2E2E}
\definecolor{LightGrey}{HTML}{666666}
\definecolor{DarkPastelRed}{HTML}{450808}
\definecolor{PastelRed}{HTML}{8F0D0D}
\definecolor{GoldenEarth}{HTML}{E7D192}
\colorlet{name}{black}
\colorlet{tagline}{PastelRed}
\colorlet{heading}{DarkPastelRed}
\colorlet{headingrule}{GoldenEarth}
\colorlet{subheading}{PastelRed}
\colorlet{accent}{PastelRed}
\colorlet{emphasis}{SlateGrey}
\colorlet{body}{LightGrey}

% Change some fonts, if necessary
\renewcommand{\namefont}{\LARGE\rmfamily\bfseries}
\renewcommand{\personalinfofont}{\footnotesize}
\renewcommand{\cvsectionfont}{\large\rmfamily\bfseries}
\renewcommand{\cvsubsectionfont}{\large\bfseries}

\newcommand\blfootnote[1]{%
  \begingroup
  \renewcommand\thefootnote{}\footnote{#1}%
  \addtocounter{footnote}{-1}%
  \endgroup
}

% Change the bullets for itemize and rating marker
% for \cvskill if you want to
\renewcommand{\itemmarker}{{\small\textbullet}}
\renewcommand{\ratingmarker}{\faCircle}

\begin{document}
\name{Thomas Bidne}
\tagline{Software Engineer}

\personalinfo{%
  % Not all of these are required!
  \email{tbidne@gmail.com}
  \homepage{tbidne.github.io}
  \linkedin{tbidne}
  \github{tbidne}
  %% You MUST add the academicons option to \documentclass, then compile with LuaLaTeX or XeLaTeX, if you want to use \orcid or other academicons commands.
  % \orcid{0000-0000-0000-0000}
  %% You can add your own arbtrary detail with
  %% \printinfo{symbol}{detail}[optional hyperlink prefix]
  % \printinfo{\faPaw}{Hey ho!}[https://example.com/]
  %% Or you can declare your own field with
  %% \NewInfoFiled{fieldname}{symbol}[optional hyperlink prefix] and use it:
  % \NewInfoField{gitlab}{\faGitlab}[https://gitlab.com/]
  % \gitlab{your_id}
}

\makecvheader
%% Depending on your tastes, you may want to make fonts of itemize environments slightly smaller
% \AtBeginEnvironment{itemize}{\small}

%% Set the left/right column width ratio to 6:4.
\columnratio{0.6}

\cvsection{About}

I am a software engineer with 10 years of professional experience. I have been writing code since I was 14, for both academic and personal reasons. I write code most days, whether it is paid work, open-source contributions, or just plain experimentation for fun.

% Start a 2-column paracol. Both the left and right columns will automatically
% break across pages if things get too long.
\begin{paracol}{2}

\cvsection{Experience}

\cvevent{Senior Software Engineer}{Platonic.Systems}{November 2020 -- Present}{}

\begin{itemize}
    \item Implemented "smart-contracts" with a high emphasis on correctness in Haskell, Nix, and PureScript.
    \item Designed and implemented a website for data visualization and machine learning in Python.
    \item Improved a client's software's robustness by re-implementing fragile JavaScript APIs in Haskell.
    \item Worked on a high-performance full-stack Haskell web application for supply-chain analysis.
\end{itemize}

\divider

\cvevent{Senior DevSecOps Engineer}{BridgePhase}{March 2016 -- November 2020}{USCIS, Washington D.C.}

\begin{itemize}
    \item Full-stack web development for the Electronic Immigration System (ELIS) application, used by United States Citizenship and Immigration (USCIS) officers for handling applicant case flows.
    \item Technologies used included Java+Spring, Angular, React, Javascript, Typescript, Oracle/Postgres, Docker, OpenShift, and CI/CD via Jenkins.
    \item Maintained a Red Hat Enterprise Linux virtual machine for testing.
    \item Participated in multiple "tiger teams" for developing cross-team solutions to shared problems.
    \item Managed daily production deployments on a semi-regular basis.
\end{itemize}

\divider

\cvevent{Software Engineer}{GBL Systems}{May 2013 -- February 2016}{NAS Pax River, Patuxent MD}
\begin{itemize}
  \item Worked on the C++ real-time distributed simulation Next Generation Threat System (NGTS) for the United States Navy.
  \item Worked on network protocols and integrating NGTS with third-party software. This included designing/implementing APIs and travelling to customer sites to provide support.
\end{itemize}

%% Switch to the right column. This will now automatically move to the second
%% page if the content is too long.
\switchcolumn

%% Yeah I didn't spend too much time making all the
%% spacing consistent... sorry. Use \smallskip, \medskip,
%% \bigskip, \vpsace etc to make ajustments.

\cvsection{Education}

\cvevent{B.Sc. in Computer Science}{University of Maryland, College Park}{2009 -- 2013}{}
\begin{itemize}
\item Minor: Astronomy
\item College Park Scholars Graduate
\end{itemize}

\cvsection{Open Source}

\begin{itemize}
    \item I have contributed to opensource via my own applications / libraries and other popular libraries, especially in the Haskell community. This includes GHC, the primary Haskell compiler.
    \item More information can be found on my \href{https://www.github.com/tbidne}{github} and \href{https://tbidne.github.io}{website}.
\end{itemize}

\cvsection{Skills}

\cvtag{Distributed Systems}
\cvtag{Networking} \\
\cvtag{Full-stack development}
\cvtag{Haskell}
\cvtag{Java}
\cvtag{JavaScript}
\cvtag{Typescript}
\cvtag{PureScript}
\cvtag{Nix}
\cvtag{Python}
\cvtag{C++}
\cvtag{Ruby}
\cvtag{Docker}
\cvtag{OpenShift}

\end{paracol}

\blfootnote{\textit{References available upon request}}

\end{document}
